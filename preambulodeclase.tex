%Preámbulo para trabajar las Clases
\usepackage[utf8]{inputenc}
\usepackage[spanish]{babel}
\setlength{\parindent}{0pt}
\setlength{\parskip}{0em}
%Para agregar paquetes adicionales, colocar coma y agregar:
\usepackage{latexsym, amssymb, amsmath, amsthm, color, pstricks-add, asymptote, graphicx, fancyhdr, caption, cancel,multicol}
%Para modificar los márgenes del documento:
\usepackage[paper=letterpaper,right=3cm,left=3cm,top=2.5cm,bottom=2.5cm]{geometry}
%Para modificar el interlineado del documento: 
\renewcommand{\baselinestretch}{1.15}

%Hacer cálculos matematicos
\usepackage{expl3, xparse}
\ExplSyntaxOn
\DeclareDocumentCommand { \myformat }{m}
  { \fp_to_decimal:n { round((#1),9) } }
\ExplSyntaxOff

%Simbolos matemáticos
\newcommand{\ndiv}{\hspace{-4pt}\not|\hspace{2pt}}

\usepackage{hyperref}
\hypersetup{
    colorlinks=true,
    linkcolor=blue,
    filecolor=magenta,      
    urlcolor=cyan,
}
 
\urlstyle{same}