\documentclass{article}
\usepackage{answers}

% Definir las asociaciones para sugerencias y soluciones
\Newassociation{hint}{HintFile}{all-hints}
\Newassociation{solu}{SolutionFile}{all-solutions}

% Configurar el formato de las sugerencias y soluciones
\renewenvironment{HintFile}[1]{\item[\textbf{#1.}]}{}
\renewenvironment{SolutionFile}[1]{\item[\textbf{#1.}]}{}

\begin{document}

% Abrir los archivos para sugerencias y soluciones
\Opensolutionfile{all-hints}
\Opensolutionfile{all-solutions}

\section{Problemas Propuestos}

Decir cuántos términos tiene la expansión de $(a+b+c)^8$ y de qué forma son.
\begin{hint}
Piensa en cada término de la expansión como un producto del tipo $a^{x_1} b^{x_2} c^{x_3}$, donde $x_1 + x_2 + x_3 = 8$.
\end{hint}
\begin{solu}
Cada término de la expansión de $(a+b+c)^8$ es de la forma $a^{x_1} b^{x_2} c^{x_3}$, donde $x_1 + x_2 + x_3 = 8$ y $x_1, x_2, x_3$ son enteros no negativos. Este es un problema de contar el número de soluciones enteras no negativas para esta ecuación, lo que equivale a:
\[
\binom{8+3-1}{3-1} = \binom{10}{2} = 45.
\]
Por lo tanto, hay 45 términos en la expansión y cada término tiene la forma $a^{x_1} b^{x_2} c^{x_3}$.
\end{solu}

% Cerrar los archivos para sugerencias y soluciones
\Closesolutionfile{all-hints}
\Closesolutionfile{all-solutions}

\section{Sugerencias}

\textbf{Sugerencias:}
\begin{enumerate}
\input{all-hints}
\end{enumerate}

\section{Soluciones}

\textbf{Soluciones:}
\begin{enumerate}
\input{all-solutions}
\end{enumerate}

\end{document}
